\documentclass[10pt,compress,mathserif]{beamer}
\usetheme{Berkeley}
\usepackage{amsmath,minibox,amssymb,amsfonts,amsthm,graphicx,color,multirow,array,tikz,hyperref}
\usetikzlibrary{arrows,snakes,backgrounds,patterns,matrix,shapes,fit,calc,shadows,positioning}

\setbeamertemplate{navigation symbols}{}
\setbeamerfont{caption}{size=\footnotesize}

\title{Control Systems Project}
\author{Hammad Ahmad\\
        Department of Computer Systems Engineering\\
        University of Engineering \& Technology, Peshawar}
\rightskip=0pt plus 0pt
\boldmath

\begin{document}

\begin{frame}
    \titlepage
\end{frame}

\section{Introduction}
\begin{frame}{Project Objectives}
    \begin{enumerate}
        \item Analyze system stability using multiple methods
        \item Evaluate system controllability and observability
        \item Implement and simulate controller responses
        \item Design and implement PID controller
        \item Analyze steady-state error performance
    \end{enumerate}
\end{frame}

\section{System Analysis}
\begin{frame}{State-Space Model}
The state-space representation of the system:
\begin{equation}
\begin{bmatrix} \dot{x_1}\\  \dot{x_2} \\  \dot{x_3}\end{bmatrix}
= \begin{bmatrix}
3 & 2 & 3 \\
3 & 2 & 4  \\
1 & 1 & 3 \end{bmatrix}
\begin{bmatrix} x_1\\  x_2 \\  x_3\end{bmatrix} +
\begin{bmatrix}
4\\
0\\
3 \end{bmatrix}
\begin{bmatrix} u_1 \end{bmatrix}
\end{equation}

\begin{equation}
y = \begin{bmatrix}
1 & 1 & 0.5
\end{bmatrix}
\begin{bmatrix} x_1 \\ x_2 \\ x_3 \end{bmatrix}
\end{equation}
\end{frame}

\section{Stability Analysis}
\begin{frame}{Eigenvalue Analysis}
System characteristics:
\begin{equation}
\text{Eigenvalues: } \lambda_1 = 6.8023, \lambda_2 = 1.3099, \lambda_3 = -0.1122
\end{equation}

\begin{equation}
\text{System poles: } p_1 = 6.8023, p_2 = 1.3099, p_3 = -0.1122
\end{equation}
\end{frame}

\begin{frame}{Routh-Hurwitz Stability Analysis}
\begin{table}[h]
\begin{center}
\begin{tabular}{|l|c|l|}
\hline
$s^3$ & 1 & 8 \\ \hline
$s^2$ & -8 & 1 \\ \hline
$s^1$ & $-\frac{1}{-8}\times\begin{vmatrix} 1 & 8\\ -8 & 1 \end{vmatrix}=8.125$ & 0 \\
$s^0$ & $-\frac{1}{8.125}\times\begin{vmatrix} -8 & 1\\ 8.125 & 0 \end{vmatrix}=1$ & 0 \\ \hline
\end{tabular}
\end{center}
\end{table}

Conclusion: System unstability confirmed by presence of second and third row.
\end{frame}

\begin{frame}{System Response Characteristics}
\begin{figure}[h!]
\centering
\includegraphics[scale=0.5]{Step_response.png}
\caption{Uncompensated System Step Response}
\end{figure}
\end{frame}

\begin{frame}{Pole-Zero Distribution}
\begin{figure}[h!]
\centering
\includegraphics[scale=0.5]{pz_map.png}
\caption{System Pole-Zero Map}
\end{figure}
\end{frame}

\begin{frame}{Root Locus Analysis}
\begin{figure}[h!]
\centering
\includegraphics[scale=0.5]{root_locus.png}
\caption{System Root Locus}
\end{figure}
\end{frame}

\section{Controllability Analysis}
\begin{frame}{System Controllability}
Controllability matrix \(\mathcal{P}\):
\[
\mathcal{P} = \begin{bmatrix}
4 & 21 & 150 \\
0 & 24 & 163 \\
3 & 13 & 84
\end{bmatrix}
\]

The system is fully controllable as \(\text{rank}(\mathcal{P}) = 3\), equal to the order of system matrix \(A\).

State feedback gain matrix:
\[
K = \begin{bmatrix}
91.2439 & 8.2683 & -106.6585
\end{bmatrix}
\]
\end{frame}

\section{Observability Analysis}
\begin{frame}{System Observability}
Observability matrix \(\mathcal{Q}\):
\[
\mathcal{Q} = \begin{bmatrix}
1.0000 & 1.0000 & 0.5000 \\
6.5000 & 4.5000 & 8.5000 \\
41.5000 & 30.5000 & 63.0000
\end{bmatrix}
\]

Full rank confirms system observability. Observer gain matrix:
\[
L = 1.0 \times 10^4 \begin{bmatrix}
-1.9481 \\
1.5723 \\
0.7911
\end{bmatrix}
\]
\end{frame}

\section{Controller Implementation}
\begin{frame}{Observer-Based Controller Performance}
\begin{figure}[h!]
\centering
\includegraphics[width=0.75\textwidth]{observed_based_feedback_controller.png}
\caption{Observer-Based Feedback Control Response}
\end{figure}
\end{frame}

\begin{frame}{PID Controller Performance}
\begin{figure}[h!]
\centering
\includegraphics[width=0.75\textwidth]{Pid_controller.png}
\caption{PID Controller System Response}
\end{figure}
\end{frame}

\section{Performance Analysis}
\begin{frame}{Controller Performance Comparison}
\begin{figure}[h!]
\centering
\includegraphics[width=0.75\textwidth]{pid_simulation.png}
\caption{PID Controller Dynamic Response}
\end{figure}
\end{frame}

\begin{frame}{Observer Performance Evaluation}
\begin{figure}[h!]
\centering
\includegraphics[width=0.75\textwidth]{observer_simualtion.png}
\caption{Observer-Based Control System Response}
\end{figure}
\end{frame}

\begin{frame}{Steady-State Error Analysis}
\begin{figure}[h!]
\centering
\includegraphics[width=0.75\textwidth]{SSE_simulation.png}
\caption{Steady-State Error Performance}
\end{figure}
\end{frame}

\end{document}