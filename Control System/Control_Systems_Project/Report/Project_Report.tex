\documentclass{article}
\usepackage{amsmath,epsfig,rotating}

\begin{document}

\begin{flushleft}
{\sc \Large Control Systems - Project Report} \hfill \\ $ $ \\ 
\medskip
Name: Hammad Ahmad \\ 
Enrollment Number: 21PWCSE1999\\ 
Section: A \\ 
\today
\end{flushleft}

\section{Problem which is considered}
a. Check the stabiliity of the system using all the methods you know. \\ 
b. Compute the controllability and observibility for the system.If the system is unstable, design a suitable controller for it. \\ 
c. Simulate tge system using controller and show all the responses.\\ 
d. Design a PID controller.\\ 
e. Compute the steady state errors before and after designing the controller.\\

% This command is done for extra space
\vskip30pt

\section{Solution}
In this report, we address the above problem and explain each subproblem in detail.

\subsection{State-space Representation of the System}
The state-space representation of the system can be written as follows:
\begin{equation}
\begin{bmatrix} 
\dot{x_1} \\  
\dot{x_2} \\  
\dot{x_3} 
\end{bmatrix}
= \begin{bmatrix}
3 & 2 & 3 \\
3 & 2 & 4 \\
1 & 1 & 3  
\end{bmatrix}
\begin{bmatrix} 
x_1 \\  
x_2 \\  
x_3  
\end{bmatrix} +
\begin{bmatrix}
4 \\ 
0 \\ 
3  
\end{bmatrix}
u
\end{equation}

\begin{equation}
y = \begin{bmatrix}
1 & 1 & 0.5
\end{bmatrix}
\begin{bmatrix} 
x_1 \\  
x_2 \\  
x_3  
\end{bmatrix}
\end{equation}

The state-space matrices are defined as follows:

\[
A = \begin{bmatrix}
3 & 2 & 3 \\
3 & 2 & 4 \\
1 & 1 & 3  
\end{bmatrix}, \quad 
B = \begin{bmatrix}
4 \\
0 \\
3
\end{bmatrix}, \quad 
C = \begin{bmatrix}
1 & 1 & 0.5
\end{bmatrix}, \quad D = 0
\]

\subsection{Stability analysis of the system}
In this section, we analyze the stability of the system using eigenvalues, poles, step response, and root locus.

The eigenvalues of the system are calculated from the matrix \( A \). The eigenvalues \(\lambda_1\), \(\lambda_2\), and \(\lambda_3\) of the system are:

\[
\lambda_1 = 6.8023, \quad \lambda_2 = 1.3099, \quad \lambda_3 = -0.1122
\]

We observe that two eigenvalues is negative, which indicates the system is unstable. 

Next, the poles of the transfer function are also computed. The poles \( p_1 \), \( p_2 \), and \( p_3 \) are:

\[
p_1 = 6.8023, \quad p_2 = 1.3099, \quad p_3 = -0.1122
\]

Since two of the poles is positive, the system is again determined to be unstable. 
\vspace{10pt}

The step response of the system is shown below. It confirms that the system is unstable as expected from the eigenvalues and poles. The plot below illustrates the response:

\begin{figure}[h!]
\centering
\includegraphics[width=0.75\textwidth]{Step_Res.png} % Ensure the image is in the same directory as this .tex file
\caption{Step Response of the system, indicating instability.}
\end{figure}

\vspace{10pt} % Adds a small gap between the figures

The performance metrics for the step response are as follows:

\[
\text{Rise Time: } \text{Tr} = 11.9286
\]
\[
\text{Transient Time: } \text{Tt} = 316.8811
\]
\[
\text{Settling Time: } \text{Ts} = 316.8811
\]
\[
\text{Settling Min: } \text{SettlingMin} = 2.1575 \times 10^{25}
\]
\[
\text{Settling Max: } \text{SettlingMax} = 2.1575 \times 10^{25}
\]
\[
\text{Overshoot: } \text{OS} = 0
\]
\[
\text{Undershoot: } \text{US} = 0
\]
\[
\text{Peak Value: } \text{Peak} = 2.1575 \times 10^{25}
\]
\[
\text{Peak Time: } \text{PeakTime} = 317
\]

These metrics indicate that the system's step response exhibits extremely high settling times and peak values, which further confirms the system's instability.

\vspace{10pt} % Adds a small gap between the figures

The root locus of the system is shown below:

\begin{figure}[h!]
\centering
\includegraphics[width=0.75\textwidth]{r_locus.png} % Ensure the image is in the same directory as this .tex file
\caption{Root Locus of the system.}
\end{figure}

\vspace{10pt} % Adds a small gap between the figures

Next, we construct a Routh-Hurwitz table to check the stability of the system:

\[
\renewcommand{\arraystretch}{1.5}
\begin{array}{|c|c|c|c|}
\hline
\textbf{Order} & \textbf{Column 1} & \textbf{Column 2} & \textbf{Column 3} \\ \hline
s^3 & 1   & 8   & 0 \\ \hline
s^2 & -8  & 1   & 0 \\ \hline
s^1 & \frac{65}{8} & 0 & 0 \\ \hline
s^0 & 1   & 0   & 0 \\ \hline
\end{array}
\]

As there are 2 sign changes in the first column, the system is unstable.


\subsection{Controllability analysis of the system}
The controllability of the system is checked using the controllability matrix. The controllability matrix \( \mathcal{C} \) is calculated as follows:

\[
\mathcal{C} = \begin{bmatrix}
4 & 21 & 150 \\
0 & 24 & 163 \\
3 & 13 & 84
\end{bmatrix}
\]

The rank of the controllability matrix is:

\[
\text{rank}(\mathcal{C}) = 3
\]

Since the rank of the controllability matrix is equal to \( n \), which is the size of the matrix \( A \) (i.e., 3), the system passes the controllability test. This means the system is controllable. 

Now, we proceed to find the state feedback gain matrix \( K \). The state feedback gain matrix \( K \) is computed as:

\[
K = \begin{bmatrix}
91.2439 & 8.2683 & -106.6585
\end{bmatrix}
\]

Since the system is controllable, we have successfully designed the state feedback controller.
\vspace{10pt} % Adds a small gap between the figures

\subsection{Observability analysis of the system}
The observability of the system is checked using the observability matrix. The observability matrix \( \mathcal{O} \) is calculated as follows:

\[
\mathcal{O} = \begin{bmatrix}
1.0000 & 1.0000 & 0.5000 \\
6.5000 & 4.5000 & 8.5000 \\
41.5000 & 30.5000 & 63.0000
\end{bmatrix}
\]

The system is observable, as the rank of the observability matrix is full, implying that the system's states can be reconstructed from the output.

Now, we proceed to design the observer. The observer gain matrix \( L \) is computed as:

\[
L = 1.0 \times 10^4 \begin{bmatrix}
-1.9481 \\
1.5723 \\
0.7911
\end{bmatrix}
\]

With the observer gain matrix \( L \) calculated, we can design the observer for the system.


\subsection{Controller Design for the system}

Since the system passed both the controllability and observability tests, we proceeded to design the observer-based feedback controller. The observer-based feedback controller utilizes the observer gain matrix \( L \) and the state feedback matrix \( K \), which were derived earlier. The controller design is implemented to stabilize the system by placing the poles at the desired locations.

The image below illustrates the response of the system with the observer-based feedback controller.

\begin{figure}[h!]
\centering
\includegraphics[width=0.75\textwidth]{observed_based_feedback_controller.png} % Ensure the image is in the same directory as this .tex file
\caption{System Response with Observer-Based Feedback Controller.}
\end{figure}

Additionally, we designed a PID controller for the system to compare its performance with the observer-based feedback controller. The PID controller is tuned to provide the desired system behavior. The image below shows the system response with the PID controller.

\begin{figure}[h!]
\centering
\includegraphics[width=0.75\textwidth]{Pid_controller.png} % Ensure the image is in the same directory as this .tex file
\caption{System Response with PID Controller.}
\end{figure}


\section{Results and Discussions}

1. \textbf{PID Controller Simulation}: The simulation of the PID controller’s performance is shown in Figure 5. This simulation demonstrates how the PID controller regulates the system, maintaining stability and achieving the desired output.

\begin{figure}[h!]
\centering
\includegraphics[width=0.75\textwidth]{pid_simulation.png} % Ensure the image is in the same directory as this .tex file
\caption{PID Controller Simulation.}
\end{figure}

2. \textbf{Observer-Based Feedback Controller Simulation}: Figure 6 presents the simulation of the observer-based feedback controller. This simulation illustrates the effectiveness of the observer in estimating system states and its impact on overall performance, improving system robustness.

\begin{figure}[h!]
\centering
\includegraphics[width=0.75\textwidth]{observer_simualtion.png} % Ensure the image is in the same directory as this .tex file
\caption{Observer-Based Feedback Controller Simulation.}
\end{figure}

3. \textbf{Steady-State Error Simulation}: The steady-state errors of the system are depicted in Figure  7. This simulation analyzes the system's response over time and demonstrates how the steady-state error is minimized with the applied controllers.

\begin{figure}[h!]
\centering
\includegraphics[width=0.75\textwidth]{SSE_simulation.png} % Ensure the image is in the same directory as this .tex file
\caption{Steady-State Error Simulation.}
\end{figure}

Each simulation offers valuable insights into the system's behavior under different control strategies and helps in evaluating their effectiveness in achieving the desired performance.



\end{document}
